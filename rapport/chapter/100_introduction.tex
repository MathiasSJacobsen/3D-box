\chapter{Introduction}

This report is part of my INF319 project at the \gls{uib}. The goal of the project was to
understand event-based \gls{gui} programming and the limitation to such programming.
Specify dataflow constraints and understand how they connect to \gls{gui} widgets.
Understand the possibilities and limitations of constraint systems based on \gls{gui}s.
The multiway dataflow constraint system used in the development of this project is
HotDrink. Before developing this project I had no prior experience with HotDrink. 
\\TODO: Skriv hva rapporten inneholder.
\newpage

\section{Constraint systems}
\label{sec:constraint-systems}
A constraint system can be seen as a tuple ${\langle V, C \rangle}$, where $V$ is a set 
of \textit{variables} and $C$ a set of \textit{constraints}. Each variable in $V$ has a 
associated value of a given type (string, integer, boolean, object, etc.). Each 
constraint in $C$ is a tuple ${\langle R, r, M \rangle}$. The variables involved in the 
constraint is given by ${R \subseteq V}$, $r$ is some \textit{n}-ary relation among 
variables in $R$, where ${n = \lvert R \rvert}$. $M$ is a set of non-empty set of 
\textit{constraint system methods}. Executing any method $m$ in $M$ enforces the 
constraint by computing values for some subset og $R$, using another disjoint subset of 
$R$ as inputs, such that the relation $R$ is satisfied~\cite{AlgorithmsForUserInterfaces}. 

\section{HotDrink}

In this project I used the multiway dataflow constraint system library 
HotDrink~\cite{HotDrink}, witch is a JavaScript-based library for multiway dataflow 
constraint systems in \gls{gui}s. Instead of writing explicit event handlers,
the programmer writes declarative specification of data dependencies,
from which the library derives the GUI behavior.
This library features a \gls{dsl} for defining constraint systems.
The \gls{dsl} allows one to specify \emph{components}, \emph{constraints}, \emph{methods} 
and \emph{variables}.
A component in HotDrink holds a set of constraints and variables, as described in 
section~\ref{sec:constraint-systems}. Variables often depend on each other,
in that case it gets into a setting of multi-way dataflow.

The HotDrink \gls{dsl} is implemented JavaScript tagged template literals
\footnote{\href{https://developer.mozilla.org/en-US/docs/Web/JavaScript/Reference/Template_literals}{Template literals}}, 
which can be seen in Listing~\ref{HotDrinkDSL}. This lets the programmer integrate the 
library with frontend JavaScript frameworks such as 
React\footnote{For more information about the framework can be found at \href{https://reactjs.org/}{reactjs.org}} and 
Svelte\footnote{For more information about the framework can be found at \href{https://svelte.dev/}{svelte.dev}}. 



\begin{lstlisting}[caption={Example of the HotDrink \gls{dsl}},label=HotDrinkDSL, language=hotdrink]
import { component } from 'hot-drink';

const comp = component`
    var f=1337, c;

    constraint c1 {
        m1(c -> f) => c * (9/5) + 32;
        m2(f -> c) => (f -32) * 5/9;
    }
`;
\end{lstlisting}

\section{Svelte}
Svelte is a JavaScript framework used for building user interfaces, just like React, 
Angular, Vue. Where some of these frameworks bulk there work in the browser, Svelte 
shifts that work onto a compile step~\cite{sveltewebsite}. Svelte is written using 
TypeScript. Instead of using virtual \gls{dom} svelte uses build time to covert the 
code into JavaScript~\cite{sveltedocs}.