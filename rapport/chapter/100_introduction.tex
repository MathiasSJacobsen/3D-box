\chapter{Introduction}

This report is part of my INF319 project at the \gls{uib}. The goal of the project was to
understand event-based \gls{gui} programming and the limitation to such programming.
Specify dataflow constraints and understand how they connect to \gls{gui} widgets.
Understand the possibilities and limitations of constraint systems based on \gls{gui}s.
The multiway dataflow constraint system used in the development of this project is
HotDrink. Before developing this project I had no prior experience with HotDrink. 
\\TODO: Skriv hva rapporten inneholder.
\newpage

\section{Constraint systems}
\label{sec:constraint-systems}
A constraint system can be seen as a tuple ${\langle V, C \rangle}$, where $V$ is a set 
of \textit{variables} and $C$ a set of \textit{constraints}. Each variable in $V$ has a 
associated value of a given type (string, integer, boolean, object, etc.). Each 
constraint in $C$ is a tuple ${\langle R, r, M \rangle}$. The variables involved in the 
constraint is given by ${R \subseteq V}$, $r$ is some \textit{n}-ary relation among 
variables in $R$, where ${n = \lvert R \rvert}$. $M$ is a set of non-empty set of 
\textit{constraint system methods}. Executing any method $m$ in $M$ enforces the 
constraint by computing values for some subset og $R$, using another disjoint subset of 
$R$ as inputs, such that the relation $R$ is satisfied. 

