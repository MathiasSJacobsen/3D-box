\chapter{Multi-way dataflow constraint system}
\label{chap:constraint-systems}

A constraint system is seen as a tuple ${\langle V, C \rangle}$, where $V$ is a set of \textit{variables} and $C$ a set of \textit{constraints}~\cite{jarvi_algorithms_2009}. Each variable in $V$ has an associated value of a given type (string, integer, boolean, object, etc.). Each constraint in $C$ is a tuple ${\langle R, r, M \rangle}$. The variables involved in the constraint is given by ${R \subseteq V}$, $r$ is some \textit{n}-ary relation among variables in $R$, where ${n = \lvert R \rvert}$. $M$ is a set of non-empty set of \textit{constraint system methods}. Executing any method $m$ in $M$ enforces the constraint by computing values for some subset of $R$, using another disjoint subset of $R$ as inputs, such that the relation $R$ is satisfied. We like to denote the inputs and outputs of the method $m$: \textit{ins(m)} and \textit{outs(m)}. The programmer must ensure that the execution of a method ensures that a constraint is satisfied --- the code of a method is to be considered as a "black box". 

constraint satisfaction methods…

A constraint is a maintainable relation between a sub-set of the component variables. It can, for instance, represent a equation such as  $E_k = \frac{1}{2}mv^2$~\cite{svartveit2021multithreaded}. When a variable changes the constraint system needs to be re-enforced.

The programs \textit{dataflow} is a graph of variables, with directed edges between variables expressing data flow or variable dependencies~\cite{stokke2020declaratively}. If variable $a$ is used in the computation of a variable $b$, then $b$ must be updated whenever $a$ is changed, thus $b$ is said to be dependent on $a$, and vice versa.

\section{HotDrink}
\label{sec:hotdrink}

In this project we used the multiway dataflow constraint system library 
HotDrink~\cite{HotDrink}, which is a JavaScript-based library for multiway dataflow constraint systems in \gls{gui}s. Instead of writing explicit event handlers, the programmer writes declarative specification of data dependencies, from which the library derives the GUI behavior.

This library features a \gls{dsl} for defining constraint systems. The \gls{dsl} allows one to specify \emph{components}, \emph{constraints}, \emph{methods} and \emph{variables}. 

In HotDrink, a variable contains a value, and the programmer is able to update this value $v$ with a new value $newValue$ by calling the $v.set(newValue)$ method. To be notified whenever $v$ changes, the programmer can subscribe to $v$ by calling the $v.subscribe(callback)$ method, where the $callback$ is a callback function that is called with the new value of $v$ whenever $v$ changes. There must be established a binding between the variable and the \gls{gui} element, so either the \gls{gui} is updated when the variable updates, or the variable is updated when the \gls{gui} is updated. A variable declaration in HotDrink \gls{dsl} is similar to variable declaration in JavaScript. The declaration of $v$ with a initial value of $1$ and $w$ with a initial value of $undefiend$ goes as follows: 

\texttt{var x = 1, w;}\\\\
The keyword \texttt{var} is followed by the name of the variable. A value is optional, and if not specified, the variable is initialized to \texttt{undefined}. 

A constraint, in HotDrink, is a set of variables specified by the constraint satisfaction methods. If we denote the constraint and set of variables to $c$ and $vs$, respectively, then the constraint satisfaction methods of $c$ enforces $c$ when it is executed. Each method $m$ has a specification that uses two subsets of $vs$ --- $ins(m) \subset vs$, $outs(m) \subset vs$ and $ins(m) \cap outs(m) = \emptyset$. It uses the $ins(m)$ as arguments to the method, and returns the value that method computes to $outs(m)$. In computer science terms, we state that we read from $ins(m)$ and writes to $outs(m)$.

A component is a collection of variables and constraints that forms an independent constraint system~\cite{svartveit2021multithreaded}. A component typically correspond to a group of \gls{gui} elements. Each variable and constraint must be defined inside a component, and by that they are also owned by that component. 

Variables often depend on each other, in that case it gets into a setting of multi-way dataflow. The HotDrink \gls{dsl} is implemented JavaScript tagged template literals \footnote{\href{https://developer.mozilla.org/en-US/docs/Web/JavaScript/Reference/Template_literals}{Template literals}}, which can be seen in Listing~\ref{HotDrinkDSL}. This lets the programmer integrate the library with frontend JavaScript frameworks such as React\footnote{\url{https://reactjs.org/}} and Svelte\footnote{\url{https://svelte.dev/}}. 

\begin{lstlisting}[caption={Example of the HotDrink \gls{dsl}},label=HotDrinkDSL, language=hotdrink]
import { component } from 'hot-drink';

const comp = component`
    var f=1337, c;

    constraint c1 {
        m1(c -> f) => c * (9/5) + 32;
        m2(f -> c) => (f -32) * 5/9;
    }
`;
\end{lstlisting}

Currently there are no integrated methods in HotDrink to bind the constraint system to frontend frameworks. The programmer have to decide on the best way to integrate the HotDrink to the frontend web application of there choosing. Listings~\ref{HotDrinkBinding},~\ref{reactiveStatement},~\ref{onMountHotdrink} shows one way of binding HotDrink with Svelte.

\begin{lstlisting}[caption={Function for binding HotDrink and Svelte variable},label=HotDrinkBinding, language=javascript]
function setHDValue<T>(HDvariable: Variable<T>, n: T) {
    if (n !== HDvariable.value) { 
        HDvariable.set(n);
    };
};
\end{lstlisting}

\begin{lstlisting}[caption={Using the reactive statement, in Svelte~\cite{sveltedocs}, to update HotDrink corresopnding value, and trigger HotDrink to enforce the constraint system},label=reactiveStatement, language=javascript]
$: {
    setHDValue(HotDrinkValue, frontendValue);
}
\end{lstlisting}

\begin{lstlisting}[caption={Using the onMount callback, in Svelte~\cite{sveltedocs}, to update the frontend value that correspont to the same value in HotDrink, when the HotDrink value changes.},label=onMountHotdrink, language=javascript]
onMount(() => {
    HotDrinkValue.subscribeValue((value: number) => frontendValue = value);
});
\end{lstlisting}

In the current sate of the library there is no easy documentation for the library. There are two ways writing HotDrink. The developer can either access the HotDrink \gls{api} seen in listing~\ref{hotdrinkapi}, or the developer can use the HotDrink \gls{dsl} seen in listing~\ref{HotDrinkDSL}. Both with there own disadvantages. 

Currently, HotDrink does not have a dedicated tool support --- i.e. \gls{ide} --- which would provide \gls{dsl}-level syntax highlighting, code validation, refactoring and debugging. Using the \gls{api} can easily lead to errors and is not recommended. On line 9 and 10 of listing~\ref{hotdrinkapi} the developer have to specify the index of the input-variables and the output-variables. In the example there are only two variables --- \texttt{c} and \texttt{f} --- thus it is not complicated to keep track of the index for the variables. In lager systems, it is easy to get lost in all the variables and there indexes.

Using the \gls{dsl} is the most used way to write HotDrink code. This is also easier to understand and more compact code then the \gls{api}, as seen in the listing~\ref{HotDrinkDSL} and \ref{hotdrinkapi}. Her both have the same \textit{celsius to fahrenheit} example. In this project we used the HotDrink \gls{dsl}. The use of the tagged template literals, as discussed, makes the programming process error-prone. Finding and fixing code errors becomes a significantly difficult process. In systems larger then the celsius to fahrenheit, it is easy to get lost in all the constraints, not knowing which variables are affected by other constraints etc. This may lead to a non-desirable behavior in the system.This together with the fact that the error messages from HotDrink when using the \gls{dsl} are not helpful to the developer. There are no way of differentiating between a missing semi-colon or a missing closing bracket. The error messages are generic error stacks. 

\begin{lstlisting}[caption={Example of how to use the HotDrink \gls{api} to simulate the relationship between fahrenheit and celsius},label=hotdrinkapi, language=JavaScript]
import { Component, Method, ConstraintSpec, MaskNone } from 'hotdrink';

// create a component and emplace some variables
export const Temp = new Component("Temp");
const c = Temp.emplaceVariable("c", 1337);
const f = Temp.emplaceVariable("f");

// create a constraint spec
const toFahrenheit = new Method(2, [0], [1], [MaskNone], (c) => (c * 9 / 5 + 32));
const toCelsius = new Method(2, [1], [0], [MaskNone], (f) => (f - 32) * 5 / 9);

const graderSpec = new ConstraintSpec([toFahrenheit, toCelsius]);

// emplace a constraint built from the constraint spec
const grader = Temp.emplaceConstraint("grader", graderSpec, [c, f]);
\end{lstlisting}
