\chapter{Implementation of a parcel sending form}

\section{Posten}
The Norwegian post-service handles packages domestically and internationally from 
and to Norway. A package can in many cases be seen as a box. Thus, we decided to make a web application that would calculate the price for sending a package with Posten. To still keep the application simple, we decided to only support packages sent domestically. Posten measures the height, depth, width and weight of the package, this is used to calculate the price of sending the package. Posten uses the price model\footnote{If any of the measures --- height, depth, width --- of the package is larger than the maximum allowed there is a special package fee of 149 NOK.} seen in table~\ref{pricingmodel} for domestic packages~\cite{postenNorgespakken}.

\begin{lstlisting}[caption={HotDrink logic for determining the price},label=hotdrinkprice, language=hotdrink]
constraint price {
    m1(w, d, h, kg -> price) => {
        if (kg <= 5) {
            if (h <= 35 && w <= 25 && d <=12) {
                return 70;
            } else if (h <= 120 && w <= 60 && d <=60) {
                return 129;
            } else {
                let spesialgodstillegg = 149;
                return 129 + spesialgodstillegg;
            }
        } else if (kg <=10) {
            if (h <= 120 && w <= 60 && d <=60) {
                return 129;
            } else {
                let spesialgodstillegg = 149;
                return 129 + spesialgodstillegg;
            }
        } else if (kg <= 25) {
            if (h <= 120 && w <= 60 && d <=60) {
                return 229;
            } else {
                let spesialgodstillegg = 149;
                return 229 + spesialgodstillegg;
            }
        } else if (kg <= 35) {
            if (h <= 120 && w <= 60 && d <=60) {
                return 299;
            } else {
                let spesialgodstillegg = 149;
                return 299 + spesialgodstillegg;
            }
        }
    }
}
\end{lstlisting}

\begin{table}[h]
    \centering
    \caption{Pricing model for domestic packages from Posten}
    \label{pricingmodel}
    \begin{tabular}{|l|cl|}
    \hline
             & \multicolumn{1}{l|}{\textless 35 x 25 x 12 cm} & \textless 120 x 60 x 60 cm    \\ \hline
    0-5 kg   & \multicolumn{1}{c|}{70 NOK}                     & \multicolumn{1}{c|}{129 NOK} \\ \hline
    0-10 kg  & \multicolumn{2}{c|}{129 NOK}                                                   \\ \hline
    10-25 kg & \multicolumn{2}{c|}{229 NOK}                                                   \\ \hline
    25-35 kg & \multicolumn{2}{c|}{299 NOK}                                                   \\ \hline
    \end{tabular}
\end{table}


\section{Svelte}
In this project, to implement the application, we use the Svelte. Svelte is a JavaScript framework used for building user interfaces, just like React, Angular, Vue, etc. Where some of these frameworks bulk there work in the browser, Svelte shifts that work onto a compile step~\cite{sveltewebsite}. Svelte is written using TypeScript. Instead of using virtual \gls{dom} Svelte uses build time to covert the code into JavaScript~\cite{sveltedocs}. 

\section{ThreeJS}
For better usability of the tool, the box is rendered in 3D using JavaScript library ThreeJS~\cite{threejs}. We use SveltThree~\cite{sveltthree} which is a component library written for Svelte using ThreeJS. 

SveltThree works by adding \textit{scene}s to an \textit{canvas} component. A scene could include a \textit{camera}, \textit{lights} and \textit{meshes}. Meshes are used to represent a object in the scene---the box or package in our case---and the camera where the view of the canvas is placed. The different types of light are there to make the object feel more natural, in the application we use two types of lights: \textit{ambient-} and \textit{directional-light}. The canvas also takes a \textit{WebGLRender} component which is used to add configurations to the renderer. The setup for SveltThree usage in our application can be seen in \autoref{sveltthreesetup}.

\begin{lstlisting}[caption={Example of a SveltThree setup},label=sveltthreesetup, language=javascript]
<Canvas let:sti w={canvasWidth*.6} h={canvasHeight * .6} interactive>

    <Scene {sti} let:scene id="scene1" props={{ background: 0x00000 } } >
        
        <PerspectiveCamera {scene} id="cam1" pos={[0, 0, 0]} lookAt={[0, 0, 0]} />
        <AmbientLight {scene} intensity={1.25} />
        <DirectionalLight {scene} pos={[3, 3, 3]} />
        <Mesh
        {scene}
        geometry={cubeGeometry}
        material={cubeMaterial}
        mat={{ roughness: 0.5, metalness: 0.5, color: 0xFF8001, }}
        pos={[-1, 0, 0]}
        rot={[.3, .4, 0]}
        scale={[1, 1, 1]} 
        />

        <Mesh
            {scene}
            geometry={sphereGeometry}
            material={sphereMaterial}
            mat={{ roughness: 0.5, metalness: 0.5, color: 0xF6E05E, }}
            pos={[$widthS*(3/4)+ ($depthS/5), 0, 0]}
            rot={[.2, .2, 0]}
            scale={[1, 1, 1]} 
        />

    </Scene>

    <WebGLRenderer
    {sti}
    sceneId="scene1"
    camId="cam1"
    config={{ antialias: true, alpha: true }} 
    />

</Canvas>
\end{lstlisting}
































\begin{comment}
\section{Early implementation}

In the early stages of implementation the idea for the application was to learn how 
to use the constraint system library HotDrink in a web application. Thus the 
application in mind was a simple \gls{gui} for a box in 3D. Her we wanted to have few 
constraints and keep the complexity of the constraints low. 

In the first iteration of the project the box had three dimensions: \textit{depth}, 
\textit{width} and \textit{height}. From these three values we calculate the value 
for the \textit{volume} of the box. See \texttt{calculateVolume} method in 
listing~\ref{metricsConstraint}. \texttt{calculateHeight}, \texttt{calculateDepth}, 
\texttt{calculateWidth} methods calculate the values for \texttt{height}, 
\texttt{depth} and \texttt{width} individually, whenever the value of \texttt{volume} 
is updated. Then HotDrink checks the \textit{priority list}, and chooses the one that 
was updated last.

\begin{lstlisting}[caption={Example of the constraint calculating the values of the box},label=metricsConstraint, language=hotdrink]
var width=1, depth=1, height=1, volume;

constraint metrics {
    calculateVolume(width, depth, height -> volume) => width * depth * height;
    calculateHeight(volume, depth, width -> height) => volume / (depth * width);
    calculateDepth(volume, width, height -> depth) => volume / (width * height);
    calculateWidth(volume, depth, height -> width) => volume / (depth * height);
}
\end{lstlisting}

HotDrink previously been used with React, too some success. Thus we decided to test it 
on Svelte. 
From listings~\ref{HotDrinkBinding},~\ref{reactiveStatement},~\ref{onMountHotdrink} 
as seen in section~\ref{sec:hotdrink} we can see how to bind HotDrink to Svelte. This 
becomes discombobulating when more values are added. Thus when we add one variable in 
HotDrink, we have to make the same corresponding variable in Svelte. Then we have to 
bind these two variables together. This is done in three steps. 

First we use the 
\texttt{onMount}-function in Svelte, to make the Svelte variable subscribe to the HotDrink 
variable using the \texttt{subscribeValue}-function. See listing~\ref{onMountHotdrink}. 
This makes it that when HotDrink enforces the constraint the Svelte variable is updated. 
The \texttt{onMount}-function is used to make sure the subscribe only happens one time, 
after the component is mounted. Now the changes propagates one way from HotDrink to Svelte. 
Thus changes from Svelte to HotDrink also needs to be propagated. For this to happen we 
use the \textit{reactive statement} in Svelte. See listing~\ref{reactiveStatement}. The 
reactive statement then call the function \texttt{setHDValue}, seen in 
listing~\ref{HotDrinkBinding}. This makes it so when a variable is updated in the \gls{gui}, 
the corresponding variable in HotDrink is updated.
\end{comment}