\chapter{HotDrink} 
in this chapter…

\section{Current state of HotDrink}
\label{sec:currentstate}

As mentioned in section \ref{sec:hotdrink}, HotDrink is a JavaScript-based library for
multiway dataflow constraint systems in \gls{gui}s. In the current sate of the library 
there is no easy documentation for the library. There are two ways writing HotDrink. 
The developer can either access the HotDrink \gls{api} seen in listing~\ref{hotdrinkapi}, 
or the developer can use the HotDrink \gls{dsl} seen in listing~\ref{HotDrinkDSL}. Both 
with there own disadvantages. 

Currently, HotDrink does not have a dedicated tool support --- i.e. \gls{ide} --- which 
would provide \gls{dsl}-level syntax highlighting, code validation, refactoring and 
debugging. Using the \gls{api} can easily lead to errors and is not recommended. On line 9 and 10 
of listing~\ref{hotdrinkapi} the developer have to specify the index of the 
input-variables and the output-variables. In the example there are only two variables 
--- \texttt{c} and \texttt{f} --- thus it is not complicated to keep track of the index 
for the variables. In lager systems, it is easy to get lost in all the variables and 
there indexes.

Using the \gls{dsl} is the most used way to write HotDrink code. This is also easier to 
understand and more compact code then the \gls{api}, as seen in the listing~\ref{HotDrinkDSL} 
and \ref{hotdrinkapi}. Her both have the same \textit{celsius to fahrenheit} example. In 
this project we used the HotDrink \gls{dsl}. The use of the tagged template literals, as 
discussed in section~\ref{sec:hotdrink}, makes the programming process error-prone. 
Finding and fixing code errors becomes a significantly difficult process. In systems 
larger then the celsius to fahrenheit, it is easy to get lost in all the constraints, 
not knowing which variables are affected by other constraints. This may lead to a 
non-desirable behavior.
This together with the fact that the error messages from HotDrink when using 
the \gls{dsl} are not helpful to the developer. There are no way of differentiating 
between a missing semi-colon or a missing closing bracket. The error messages are 
generic error stacks. 

\begin{lstlisting}[caption={Example of how to use the HotDrink \gls{api} to simulate the corranltion between fahrenheit and celsius},label=hotdrinkapi, language=JavaScript]
import { Component, Method, ConstraintSpec, MaskNone } from 'hotdrink';

// create a component and emplace some variables
export const Temp = new Component("Temp");
const c = Temp.emplaceVariable("c", 1337);
const f = Temp.emplaceVariable("f");

// create a constraint spec
const toFahrenheit = new Method(2, [0], [1], [MaskNone], (c) => (c * 9 / 5 + 32));
const toCelsius = new Method(2, [1], [0], [MaskNone], (f) => (f - 32) * 5 / 9);

const graderSpec = new ConstraintSpec([toFahrenheit, toCelsius]);

// emplace a constraint built from the constraint spec
const grader = Temp.emplaceConstraint("grader", graderSpec, [c, f]);
\end{lstlisting}

\section{Current work on HotDrink}
We are currently developing tool support for HotDrink. The development tool is intended 
to be a implementation of the HotDrink \gls{dsl} as a plugin for Visual Studio Code. 
This plugin should support \gls{ide} features mentioned in section~\ref{sec:currentstate}. 
We intend to pay a special attention to implementing the debugger functionality, such as 
the dataflow in a constraint system. We also intend to look at various views and 
visualization of running constraint systems solver, such as: 
\begin{itemize}
    \item showing the current variable values in the constraint system;
    \item showing the current dataflow, with which we mean to highlight which method in the dataflow is currently being executed;
    \item showing the history of previous dataflows;
    \item showing the generation graph, to visualize the entire history of values and how certain values have been used to compute new values;
    \item showing whether the constraints are active or not.
\end{itemize}

We use the language workbench \textbf{Eclipse Xtext}~\cite{eysholdt_xtext:_2010} / 
\textbf{Langium} to implement the \gls{dsl}. From this \gls{dsl} we get syntax 
highlighted \textit{keywords}, and also autocompletion of these keywords, as seen in figure~\ref{trenger bilde}. 

