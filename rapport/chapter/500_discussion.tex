\chapter{Discussion}
\label{chap:discussion}

Graphical User Interfaces har difficult to implement correctly, and often lead to bugs and a cluster of dependencies. This has been the case for a long time, to the point where the user now expects \gls{gui}s to contain error. These issues are frequently caused by the fact that current \gls{gui} programmers manage all user interactions using event driven programming paradigms~\cite{HotDrink}. Due to the asynchronous nature of this approach, it is difficult to ensure that the same result is obtained from the same amount of sequences, but at different times. By letting the constraint system handle the coordinating asynchronous updates, we release the programmer from this error-prone task. The constraint system updates the \gls{gui}, even when some calculations fail, the resultant reactive program assures consistent outputs for any sequence of editing actions, despite the numerous conceivable interleaving of events and computations.

I have discovered how a constraint system, like HotDrink, might be a useful alternative to event driven programming. Constraint systems make it easier to deal with significant changes in business logic, along with more readable code. I have implemented a parcel sending form for parcels sent by the Norwegian post-service domestically. Compared to Posten's own solution, our approach features; a field for the dimensions \textit{height, width, depth}, the \textit{weight} and \textit{volume} of the package and the \textit{price} for sending the package. As a extra feature the user are also able to see the package in the given dimensions in a 3D-view. The 3D-view also contains a tennis-ball, with a radius of 3.4, as a real life comparison object. A change in any of the dimensions, height, width or depth, will cause a change to the volume etc. 

HotDrink can be integrated with various web frameworks, such as React, Svelte, etc. When using the library the developer, as discussed, is expected to know the syntax of the \gls{dsl} embedded into HotDrink, and this might prevent wider adoption of the library. To mitigate this, we are developing currently tool support for HotDrink. The development tool is intended to be a implementation of the HotDrink \gls{dsl} as a plugin for Visual Studio Code. This plugin should support \gls{ide} features mentioned in section~\ref{sec:hotdrink}.We intend to pay a special attention to implementing the debugger functionality, such as 
the dataflow in a constraint system. We also intend to look at various views and 
visualization of running constraint systems solver, such as: 
\begin{itemize}
    \item showing the current variable values in the constraint system;
    \item showing the current dataflow, with which we mean to highlight which method in the dataflow is currently being executed;
    \item showing the history of previous dataflows;
    \item showing the generation graph, to visualize the entire history of values and how certain values have been used to compute new values;
    \item showing whether the constraints are active or not.
\end{itemize}

We use the language workbench \textbf{Eclipse Xtext}~\cite{eysholdt_xtext:_2010} / 
\textbf{Langium} to implement the \gls{dsl}. From this \gls{dsl} we get syntax 
highlighted \textit{keywords}, and also autocompletion of these keywords. 
